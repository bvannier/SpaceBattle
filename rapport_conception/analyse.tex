\chapter{Analyse conceptuelle}

L’étude UML\footnote{Unfied Modeling Language} nous a permis de modéliser l’application par abstraction des détails inutiles pour la simplifier. Elle nous a également aidé à clarifier les besoins du jeu et les spécificités de l’application afin de la développer. Nous nous sommes concentrés sur différents diagrammes UML qui nous semblaient les plus adaptés à notre besoin. Nous avons donc choisi de modéliser notre projet à travers des diagrammes de cas d’utilisation, un diagramme d’activité, des diagrammes d’états-transitions, un diagramme de séquence, un diagramme de composants et un diagramme de classe.

\section{Diagrammes de cas d'utilisation}

Le diagramme de cas d’utilisation permet la représentation de l’implication de l’utilisateur dans l’application et des interactions entre l’utilisateur et le système. On peut ainsi décrire ce que le futur système devra faire, sans spécifier de quelle manière il le fera.
Pour notre projet, nous avons deux diagrammes de cas d’utilisation.

Le premier diagramme représente le lancement du jeu et donc l’implication de l’utilisateur dans l’application pour initialiser une partie.\\

Le second diagramme représente le joueur en pleine action, c'est-à-dire durant une partie. Les différentes actions qu’il pourra effectuer son directement liées aux principes du jeu et les relations d’inclusion et d’exclusion permettent d’identifier les enchainements et les interactions avec le système.

\section{Diagramme d'activité}

Le diagramme d’activité permet la représentation du processus d’un tour. On peut ainsi voir toutes les possibilités et tous les cas possibles offerts aux joueurs lors d’un seul tour.

\section{Diagramme d'états-transitions}

Afin de clarifier le déroulement d’un tour effectué par un jour, nous allons décomposer les différentes étapes et possibilités à travers différents diagrammes d’états-transitions.

\section{Diagramme de séquence}

Le premier rectangle symbolise la Vue, représentant l’IHM (Interface Homme Machine). Cette interface permet de faire communiquer les actions choisies par l’utilisateur (lancé de dés, choix de cases) au contrôleur (la partie). Le contrôleur du jeu récupère les données envoyées par la vue afin d'exécuter les traitements, et s'assurer que les règles du jeu soient respectées. Une fois que le contrôleur a effectué les traitements nécessaires, il envoie le résultat aux modèles. Le second rectangle représente le contrôleur (Partie), c'est lui qui va gérer l'ensemble du jeu, comme par exemple le lancement d'une partie. Les autres rectangles symbolisent les modèles. Les  lignes verticales pointillées représentent les lignes de vie des objets. Les actions à faire sont énumérées et symbolisées par des  guillemets.

\section{Diagramme de classes}

\section{Diagramme de composants}

Le diagramme de composants décrit l'organisation du système du point de vue des éléments logiciels comme les modules (paquetages, fichiers sources, bibliothèques, exécutables), des données (fichiers, bases de données) ou encore d'éléments de configuration (paramètres, scripts, fichiers de commandes). Ce diagramme permet de mettre en évidence les dépendances entre les composants (qui utilise quoi).