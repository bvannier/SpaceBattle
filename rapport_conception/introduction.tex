\chapter*{Introduction\markboth{Introduction}{}}
\addcontentsline{toc}{chapter}{Introduction}

Space Battle est un jeu de société qui dispose d’un plateau de jeu représentant une partie du système solaire Alpha du Centaure composé de deux astéroïdes, quatre stations orbitales de quatre couleurs différentes situées aux quatre coins du plateau, quatre cases « artefact », quatre cases « station d'énergie ». La boîte de jeu contient également deux dés à six faces, douze pions (trois vaisseaux par joueur : un vaisseau-mère, un battlecruiser et une capsule de survie) et quatorze artefacts.

Chaque joueur est le commandant d'une flotte. Il choisit la couleur de sa station orbitale et de sa flotte. A quatre joueurs, chacun prend une flotte. A trois joueurs, une flotte est laissée de côté. A deux joueurs, chacun prend deux flottes. Une flotte est un groupe de trois vaisseaux : vaisseau-mère, battlecruiser et capsule de survie.

Chaque joueur part pour la course aux artefacts.  Sur le plateau de jeu, il y a quatre cases oranges. Ce sont les cases « artefact ». Le but du jeu réside dans le fait d'atteindre l'une de ces cases, prendre un artefact, l'emporter avec son vaisseau et le ramener à sa station orbitale. Pour gagner, il faut être le premier à réussir cette prise trois fois. Nous pouvons également noter la présence de quatre cases grises. Ce sont les cases « station d'énergie ». A partir de ces cases, un joueur recharge son vaisseau en énergie ce qui lui permet dans les tours suivants de viser et tirer sur l'un de ses adversaires depuis n'importe où sur la carte.

Le joueur commence à jouer avec son vaisseau-mère. Si ce dernier se fait détruire à la suite du tir d'un adversaire, le vaisseau-mère est remplacé par le battlecruiser. Enfin si  le battlecruiser est détruit, le joueur doit rentrer à sa station orbitale avec sa capsule de survie. Une fois rentré à la station, il lui sera alors possible de repartir avec son vaisseau-mère.\\

Notre projet est de modéliser, en C++, le jeu de société se basant sur les règles principales énoncées précédemment. Pour y arriver, la première étape consiste en la modélisation de notre future application. Ainsi, dans ce rapport de conception, nous allons vous présenter cette modélisation grâce à divers diagrammes UML.