\documentclass[french,11pt,openany,titlepage]{book} %,twoside, openright % openany -> supression page blanche
\usepackage[utf8]{inputenc}
\usepackage[headings]{fullpage}
\usepackage[french]{babel}
\usepackage{graphicx}
\usepackage[hyphens]{url}
\usepackage[T1]{fontenc}
\usepackage{fancyhdr}
\usepackage{ifthen}
\usepackage{cite}
\usepackage{url}
\usepackage{eurosym}
\usepackage{rotating}
\usepackage{listings}
\lstset{
language=C,
basicstyle=\footnotesize,
}

%==============================================================
%=============== Page de titre ================================
%==============================================================

%--------------- Package arguments -----------------------%
\usepackage[]{simpletitlepage}
%---------------------------------------------------------%

%--------------- Usual title page setup ------------------%
\title{Rapport de Conception}
\author{ Nicolas \textsc{Maloeuvre} \\Bastien \textsc{Vannier}}
%---------------------------------------------------------%

%----------------- Custom Title Page setup ---------------%
\simplesubtitle{Jeu du Space Battle}
\simpleschool{INSA Rennes}
\simpleinfo{Département \textsc{Informatique}, 4\up{ème} année}
\simplesupervisors{}
%---------------------------------------------------------%

%==============================================================
%=============== Configuration ================================
%==============================================================

%--------------------- Interligne 1.5 -------------------------

\renewcommand{\baselinestretch}{1}

%------------- Page vide après chaque chapitre ----------------

%\makeatletter
%\def\cleardoublepage{\clearpage\if@twoside \ifodd\c@page\else
%\hbox{}
%\thispagestyle{empty}
%\newpage
%\if@twocolumn\hbox{}\newpage\fi\fi\fi}
%\makeatother

\begin{document}
\frontmatter

\mainmatter
\maketitle

%-------------------- en-têtes ---------------------------------

\pagestyle{fancy} % style d'en-tête

% syntaxes en-tête :
\renewcommand{\chaptermark}[1]{\markboth{\chaptername\ \thechapter.\ #1}{}} % Nom du préambule
\renewcommand{\sectionmark}[1]{\markright{\thesection\ #1}} % X.X Nom de section

\fancyhf{} % supprime les en-têtes et pieds
%\fancyhead[LE,RO]{\thepage} % Left Even, Right Odd
%\fancyhead[LO]{\rightmark} % Left Odd
%\fancyhead[RE]{\leftmark} % Right Even
%\renewcommand{\headrulewidth}{0.3pt} % filet en haut de page
\fancyhead[LE,RO]{\thepage} % Left Even, Right Odd
\fancyhead[LO]{\rightmark} % Left Odd
\fancyhead[RE]{\leftmark} % Right Even
\addtolength{\headheight}{0.5pt} % espace pour le filet
\renewcommand{\headrulewidth}{0.3pt}
\renewcommand{\footrulewidth}{0pt} %Filet en bas

% Entete page des chapitres
\fancypagestyle{plain}{
\fancyhf{}% on efface tout 
\fancyfoot[C]{\thepage} % numéro en bas de la page 
% on efface tous les traits 
\renewcommand{\headrulewidth}{0pt}
\renewcommand{\footrulewidth}{0pt}
}

%==============================================================
%=============== Préambules ===================================
%==============================================================

%\addtolength{\voffset}{-10\baselineskip}
%\addtolength{\textheight}{10\baselineskip}
%\tableofcontents
%\addtolength{\voffset}{10\baselineskip}
%\addtolength{\textheight}{-10\baselineskip}

\begingroup\makeatletter
\def\@makeschapterhead#1{%
  \vspace*{0pt}% <---- à réduire pour avoir plus de place
  {\parindent \z@ \raggedright
    \normalfont
    \interlinepenalty\@M
    \Huge \bfseries  #1\par\nobreak
    \vskip 0pt% <---- à réduire pour avoir plus de place
  }}\makeatother
\tableofcontents
\endgroup


\chapter*{Introduction\markboth{Introduction}{}}
\addcontentsline{toc}{chapter}{Introduction}

Space Battle est un jeu de société qui dispose d’un plateau de jeu représentant une partie du système solaire Alpha du Centaure composé de deux astéroïdes, quatre stations orbitales de quatre couleurs différentes situées aux quatre coins du plateau, quatre cases « artefact », quatre cases « station d'énergie ». La boîte de jeu contient également deux dés à six faces, douze pions (trois vaisseaux par joueur : un vaisseau-mère, un battlecruiser et une capsule de survie) et quatorze artefacts.

Chaque joueur est le commandant d'une flotte. Il choisit la couleur de sa station orbitale et de sa flotte. A quatre joueurs, chacun prend une flotte. A trois joueurs, une flotte est laissée de côté. A deux joueurs, chacun prend deux flottes. Une flotte est un groupe de trois vaisseaux : vaisseau-mère, battlecruiser et capsule de survie.

Chaque joueur part pour la course aux artefacts.  Sur le plateau de jeu, il y a quatre cases oranges. Ce sont les cases « artefact ». Le but du jeu réside dans le fait d'atteindre l'une de ces cases, prendre un artefact, l'emporter avec son vaisseau et le ramener à sa station orbitale. Pour gagner, il faut être le premier à réussir cette prise trois fois. Nous pouvons également noter la présence de quatre cases grises. Ce sont les cases « station d'énergie ». A partir de ces cases, un joueur recharge son vaisseau en énergie ce qui lui permet dans les tours suivants de viser et tirer sur l'un de ses adversaires depuis n'importe où sur la carte.

Le joueur commence à jouer avec son vaisseau-mère. Si ce dernier se fait détruire à la suite du tir d'un adversaire, le vaisseau-mère est remplacé par le battlecruiser. Enfin si  le battlecruiser est détruit, le joueur doit rentrer à sa station orbitale avec sa capsule de survie. Une fois rentré à la station, il lui sera alors possible de repartir avec son vaisseau-mère.\\

Notre projet est de modéliser, en C++, le jeu de société se basant sur les règles principales énoncées précédemment. Pour y arriver, la première étape consiste en la modélisation de notre future application. Ainsi, dans ce rapport de conception, nous allons vous présenter cette modélisation grâce à divers diagrammes UML.

%==============================================================
%=============== Corps du texte ===============================
%==============================================================

\chapter{Analyse conceptuelle}

L’étude UML\footnote{Unfied Modeling Language} nous a permis de modéliser l’application par abstraction des détails inutiles pour la simplifier. Elle nous a également aidé à clarifier les besoins du jeu et les spécificités de l’application afin de la développer. Nous nous sommes concentrés sur différents diagrammes UML qui nous semblaient les plus adaptés à notre besoin. Nous avons donc choisi de modéliser notre projet à travers des diagrammes de cas d’utilisation, un diagramme d’activité, des diagrammes d’états-transitions, un diagramme de séquence, un diagramme de composants et un diagramme de classe.

\section{Diagrammes de cas d'utilisation}

Le diagramme de cas d’utilisation permet la représentation de l’implication de l’utilisateur dans l’application et des interactions entre l’utilisateur et le système. On peut ainsi décrire ce que le futur système devra faire, sans spécifier de quelle manière il le fera.
Pour notre projet, nous avons deux diagrammes de cas d’utilisation.

Le premier diagramme représente le lancement du jeu et donc l’implication de l’utilisateur dans l’application pour initialiser une partie.\\

Le second diagramme représente le joueur en pleine action, c'est-à-dire durant une partie. Les différentes actions qu’il pourra effectuer son directement liées aux principes du jeu et les relations d’inclusion et d’exclusion permettent d’identifier les enchainements et les interactions avec le système.

\section{Diagramme d'activité}

Le diagramme d’activité permet la représentation du processus d’un tour. On peut ainsi voir toutes les possibilités et tous les cas possibles offerts aux joueurs lors d’un seul tour.

\section{Diagramme d'états-transitions}

Afin de clarifier le déroulement d’un tour effectué par un jour, nous allons décomposer les différentes étapes et possibilités à travers différents diagrammes d’états-transitions.

\section{Diagramme de séquence}

Le premier rectangle symbolise la Vue, représentant l’IHM (Interface Homme Machine). Cette interface permet de faire communiquer les actions choisies par l’utilisateur (lancé de dés, choix de cases) au contrôleur (la partie). Le contrôleur du jeu récupère les données envoyées par la vue afin d'exécuter les traitements, et s'assurer que les règles du jeu soient respectées. Une fois que le contrôleur a effectué les traitements nécessaires, il envoie le résultat aux modèles. Le second rectangle représente le contrôleur (Partie), c'est lui qui va gérer l'ensemble du jeu, comme par exemple le lancement d'une partie. Les autres rectangles symbolisent les modèles. Les  lignes verticales pointillées représentent les lignes de vie des objets. Les actions à faire sont énumérées et symbolisées par des  guillemets.

\section{Diagramme de classes}

\section{Diagramme de composants}

Le diagramme de composants décrit l'organisation du système du point de vue des éléments logiciels comme les modules (paquetages, fichiers sources, bibliothèques, exécutables), des données (fichiers, bases de données) ou encore d'éléments de configuration (paramètres, scripts, fichiers de commandes). Ce diagramme permet de mettre en évidence les dépendances entre les composants (qui utilise quoi).
\chapter{Analyse des choix}

\section{Choix des classes}

\section{Choix des designs patterns}

\input{conclusion}

%==============================================================
%=============== Annexes ======================================
%==============================================================

%\input{annexes}

%==============================================================
%=============== Epilogues ====================================
%==============================================================

%\input{glossaire}
%\input{bibliographie}
%\input{resume}

\end{document}
